% !TeX root = RJwrapper.tex
\title{A computational analysis of R programming style variations in the last
20 years based on 94 million lines of code from all CRAN packages}
\author{by Chung-hong Chan, Chia-Yi Yen, Mia Huai-Wen Chang}

\maketitle

\abstract{%
The coexistence of multiple programming styles confuses new users and
makes enforcing best practice difficult. This problem is aggravated by
the lack of a universally accepted style guide in the R community. To
investigate that, we quantified the programming style variation (PSV) in
all CRAN packages and observed an emerging consensus in style since
2016, as indicated by the dampened increasing trend in PSV. It seems
that a new consensus-based best practice is forming, which is a mixture
of various R style guides. Concretely, we summarized the ``ins \& outs''
of different styles based on popularity across years (e.g., rapid rise
of underscore\_fun\_name and fall of dotted.fun.name since 2013) and
pointed out the least agreed style elements (e.g., -\textgreater{} v.s.
=, space after a comma). Moreover, we identified a source of PSV (the
``Naughty, Naughty!'') by looking into the style differences between
clusters of related packages (e.g., Finance v.s. Biostatistics). Our
analysis raises an open question to all stakeholders of the R community,
i.e., the R Foundation, opinion leaders, package developers, and
ordinary users: should we adopt an official R style guide as in the case
of Python's PEP8? The findings from this study validate the R
community's effort in reducing PSV and suggest future directions.
}

% Any extra LaTeX you need in the preamble

\hypertarget{introduction}{%
\subsection{Introduction}\label{introduction}}

Introductory section which may include references in parentheses
\citep{R}, or cite a reference such as \citet{R} in the text.

\hypertarget{section-title-in-sentence-case}{%
\subsection{Section title in sentence
case}\label{section-title-in-sentence-case}}

This section may contain a figure such as Figure \ref{figure:rlogo}.

\begin{figure}[htbp]
  \centering
  \includegraphics{Rlogo}
  \caption{The logo of R.}
  \label{figure:rlogo}
\end{figure}

\hypertarget{method}{%
\subsection{Method}\label{method}}

\hypertarget{cran-packages}{%
\subsubsection{CRAN packages}\label{cran-packages}}

In Jaunary 2019, a static snapshot of CRAN was archived using the
\textbf{rsync} method outlined in the CRAN mirror HOWTO/FAQ guide. All
CRAN submissions from 1998 to 2018, including active, archived and
delisted packages were included for analysis.

The H1 of this study is to analyze the time-related changes in
programming style. To this end, we cannot analyze all submissions from
our static snapshot. If doing so the analysis would biased towards
packages with many submissions. In order to balance the broadness of
inclusion and bias, we sampled CRAN submissions using the ``one
submission per year'' approach. For a package, if it has multiple CRAN
submissions in a given year, only one submission is randomly selected.

The year of publication of a package is determined by the file time
stamp of the package's tarball. This information was extracted with the
\textbf{fs} package.

\hypertarget{language-feature-extraction}{%
\subsubsection{Language feature
extraction}\label{language-feature-extraction}}

\hypertarget{summary}{%
\subsection{Summary}\label{summary}}

This file is only a basic article template. For full details of
\emph{The R Journal} style and information on how to prepare your
article for submission, see the
\href{https://journal.r-project.org/share/author-guide.pdf}{Instructions
for Authors}.

\bibliography{RJreferences}


\address{%
Chung-hong Chan\\
Mannheimer Zentrum für Europäische Sozialforschung\\
line 1\\ line 2\\
}
\href{mailto:chung-hong.chan@mzes.uni-mannheim.de}{\nolinkurl{chung-hong.chan@mzes.uni-mannheim.de}}

\address{%
Chia-Yi Yen\\
Graduate School of Economic and Social Sciences, Universität Mannheim,
Germany\\
line 1\\ line 2\\
}
\href{mailto:author2@work}{\nolinkurl{author2@work}}

\address{%
Mia Huai-Wen Chang\\
Akelius Residential Property AB, Berlin, Germany\\
line 1\\ line 2\\
}
\href{mailto:author2@work}{\nolinkurl{author2@work}}

